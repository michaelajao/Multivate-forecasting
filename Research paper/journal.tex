\documentclass[lettersize, journal]{IEEEtran}
\usepackage[utf8]{inputenc} 
\usepackage[T1]{fontenc}   
\usepackage{mathptmx}       
\usepackage{graphicx}      
\usepackage{float}          
\usepackage{algorithmic}
\usepackage{algorithm}
\usepackage{caption}        
\usepackage{subcaption}     
\usepackage{biblatex}       
\usepackage{amsmath, amsfonts, amssymb}  
\usepackage{hyperref}       

\addbibresource{reference.bib} 



\begin{document}

% Paper title
\title{Deep Learning Approaches to Multivariate COVID-19 Forecasting: A UK Case Study}
\author{Michael Ajao-olarinoye}

\maketitle
\thispagestyle{empty}


\begin{abstract}
    This document describes the most common article elements and how to use the IEEEtran class with \LaTeX \ to produce files that are suitable for submission to the IEEE.  IEEEtran can produce conference, journal, and technical note (correspondence) papers with a suitable choice of class options.
\end{abstract}

\begin{IEEEkeywords}
    Article submission, IEEE, IEEEtran, journal, \LaTeX, paper, template, typesetting.
\end{IEEEkeywords}

\section{Introduction}
\IEEEPARstart{T}{his} introduce your topic and the purpose of your paper.


\printbibliography

\end{document}
